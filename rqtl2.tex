\documentclass[12pt,t,aspectratio=169]{beamer}
\usepackage{graphicx}
\setbeameroption{hide notes}
\setbeamertemplate{note page}[plain]
\usepackage{listings}

\input{header.tex}

%%%%%%%%%%%%%%%%%%%%%%%%%%%%%%%%%%%%%%%%%%%%%%%%%%%%%%%%%%%%%%%%%%%%%%
% end of header
%%%%%%%%%%%%%%%%%%%%%%%%%%%%%%%%%%%%%%%%%%%%%%%%%%%%%%%%%%%%%%%%%%%%%%

% title info
\title{R/qtl2}
\subtitle{QTL mapping in multi-parent populations}
\author{\href{https://kbroman.org}{Karl Broman}}
\institute{Biostatistics \& Medical Informatics, UW{\textendash}Madison}
\date{\href{https://kbroman.org}{\tt \scriptsize \color{foreground} kbroman.org}
\\[-4pt]
\href{https://github.com/kbroman}{\tt \scriptsize \color{foreground} github.com/kbroman}
\\[-4pt]
\href{https://twitter.com/kwbroman}{\tt \scriptsize \color{foreground} @kwbroman}
\\[2pt]
\scriptsize {\lolit Slides:} \href{https://bit.ly/rstudio2019}{\tt \scriptsize
  \color{foreground} bit.ly/rstudio2019}
}


\begin{document}

% title slide
{
\setbeamertemplate{footline}{} % no page number here
\frame{
  \titlepage

  \vfill \hfill \includegraphics[height=6mm]{Figs/cc-zero.png} \vspace*{-1cm}

  \note{These are slides for a talk that I will give at the rstudio::conf
    on 17 January 2019.

    Source: {\tt https://github.com/kbroman/Talk\_RStudio2019} \\
    Slides: {\tt https://bit.ly/rstudio2019}
}
} }



\begin{frame}[c]{18 years of R/qtl}

\figw{Figs/rqtl_lines_code.pdf}{1.0}

\end{frame}




\begin{frame}[c]{}

\vspace*{-1mm} \hspace*{-2mm}
\figw{Figs/inbredmice.jpg}{1.2}

\end{frame}


\begin{frame}{}

\vspace*{18mm}

\centerline{
\begin{minipage}[t]{50mm}
\includegraphics[height=50mm]{Figs/da-vinci-man.jpg}
\end{minipage}
\hspace{15mm}
\begin{minipage}[t]{50mm}
\includegraphics[height=50mm]{Figs/vitruvian_mouse.jpg}
\hspace{5mm}
\href{http://daviddeen.com}{\scriptsize \lolit \tt daviddeen.com}
\end{minipage}
}
\end{frame}


\begin{frame}[c]{Intercross}
\figw{Figs/intercross.pdf}{1.0}
\end{frame}

\begin{frame}[c]{Data}

\figh{Figs/data_fig.png}{1.0}
\end{frame}





\begin{frame}[c]{QTL mapping}

\vspace{5mm}
\figw{Figs/lodcurve_insulin_with_effects.pdf}{1.0}
\end{frame}


\begin{frame}[c]{18 years of R/qtl}

\figw{Figs/rqtl_lines_code.pdf}{1.0}

\end{frame}




\begin{frame}[c]{}

\centerline{\Large Good things}

\vspace{4mm}

\onslide<2>{
\begin{itemize}
\lolit
  \item some of the code
  \item basics of the user interface
  \item diagnostics and data visualization
  \item quite comprehensive
  \item quite flexible
\end{itemize}
}

\end{frame}


\begin{frame}{}

\vspace*{16.7mm}

\centerline{\Large Bad things}

\end{frame}


\begin{frame}[c]{Input file}

\figh{Figs/datafile.pdf}{0.8}

\end{frame}


\begin{frame}[c,fragile]{Stupidest code ever}

\begin{center}
\begin{minipage}[c]{9.3cm}
\begin{semiverbatim}
\lstset{basicstyle=\normalsize}
\begin{lstlisting}[linewidth=9.3cm]
n <- ncol(data)
temp <- rep(FALSE,n)
for(i in 1:n) {
  temp[i] <- all(data[2,1:i]=="")
  if(!temp[i]) break
}
if(!any(temp)) stop("...")
n.phe <- max((1:n)[temp])
\end{lstlisting}
\end{semiverbatim}
\end{minipage}
\end{center}

\vspace{3mm}

\hfill \href{https://kbroman.org/blog/2011/08/17/the-stupidest-r-code-ever}{\scriptsize \lolit \tt kbroman.org/blog/2011/08/17/the-stupidest-r-code-ever}

\end{frame}


\begin{frame}[c]{}

  \large

  {\hilit Open source} {\lolit means}

  everyone can see my stupid mistakes

  \bigskip \bigskip \bigskip

  \onslide<2>{
    {\hilit Version control} {\lolit means}

  everyone can see every stupid mistake I've ever made
}

\end{frame}


\begin{frame}[c]{More typically bad code}

  \large
      The {\hilit \tt scantwo()} function is {\hilit 1446 lines} long.

      \bigskip \bigskip

      The related C code is 20\% of the C code in R/qtl.


\end{frame}


\begin{frame}[c]{Tests}
\end{frame}


\begin{frame}[c]{}

\begin{center}
\large

I don't need tests; I have users.
\end{center}

\bigskip

\bigskip

\hfill
{\lolit
{\textendash} \href{https://kbroman.org}{Me} \hspace*{17mm}
}

\end{frame}


\begin{frame}[c]{}

\begin{center}
\large

If you use software that lacks automated tests, \\[8pt]
you are the tests.
\end{center}

\bigskip

\bigskip

\hfill
{\lolit
{\textendash} \href{https://twitter.com/JennyBryan}{Jenny Bryan}
}

\end{frame}


\begin{frame}[c]{Baroque data structures}

  \large
      {\tt attr(mycross\$geno[["X"]]\$probs, "map")}


\end{frame}





\begin{frame}{}

\vspace*{16.7mm}

\centerline{\Large Documentation}

\end{frame}



\begin{frame}{}

\vspace*{16.7mm}

\centerline{\Large User support}

\end{frame}



\begin{frame}[c]{QTL mapping}

\vspace{5mm}
\figw{Figs/lodcurve_insulin_with_effects.pdf}{1.0}
\end{frame}



\begin{frame}[c]{Congenic line}

\figw{Figs/congenic.pdf}{1.0}

\end{frame}




\begin{frame}[c]{Advanced intercross lines}

  \figh{Figs/ail.pdf}{0.9}

\end{frame}





\begin{frame}[c]{Heterogeneous stock}

  \vspace{2mm}

  \figh{Figs/hs.pdf}{0.9}

\end{frame}


\begin{frame}[c]{Genome-scale phenotypes}

\vspace*{5mm}

\figh{Figs/mouse_on_chips.png}{0.75}
\hfill
\href{https://biochem.wisc.edu/faculty/attie}{\scriptsize \lolit Alan
  Attie} \hspace{8mm}

\end{frame}





\begin{frame}[c]{}

  \vspace*{5mm}

\figh{Figs/rqtl2_3d.png}{0.85}


\vspace{3mm}

\hfill \href{https://ropenscilabs.github.io/miner_book}{\scriptsize \lolit \tt ropenscilabs.github.io/miner\_book}

\end{frame}



\begin{frame}[c]{R/qtl2}

\vspace*{-16.2mm}

  \vspace{21mm}

  \bbi
\item High-density genotypes
\item High-dimensional phenotypes
\item Multi-parent populations
\item Linear mixed models
  \ei

  \vspace{25mm}

\hfill \href{https://kbroman.org/qtl2}{\small \tt kbroman.org/qtl2}

\end{frame}



\begin{frame}[c]{R/qtl2: \color{foreground} Let's not make the same mistakes}

  \bbi
\only<1>{
\item C++ and Rcpp
\item Roxygen2 for documentation
\item Unit tests
\item A single ``switch'' for cross type
}
\only<2>{
{\lolit
\item C++ and Rcpp
\item Roxygen2 for documentation
\item Unit tests
\item A single ``switch'' for cross type
}
}
\onslide<2>{
\item Yet another data input format
\item Flatter data structures, but still complex
}
\ei

\end{frame}



\begin{frame}{}

\vspace*{16.7mm}

\centerline{\Large Sustainable academic software}

\end{frame}







\begin{frame}[c]{Acknowledgments}

\begin{columns}[T]
  \begin{column}[T]{0.5\textwidth}
    \vspace{0pt}
\bi
\item[] Danny Arends
\item[] Gary Churchill
\item[] Nick Furlotte
\item[] Dan Gatti
\item[] Ritsert Jansen
\item[] Pjotr Prins
\item[] \'Saunak Sen
\item[] Petr Simecek
\item[] Artem Tarasov
\item[] Hao Wu
\item[] Brian Yandell
  \ei
  \end{column} \hfill
\begin{column}[T]{0.5\textwidth}
\vspace*{0mm}

  \bi
\item[] Robert Corty
\item[] Timoth\'ee Flutre
\item[] Lars Ronnegard
\item[] Rohan Shah
\item[] Laura Shannon
\item[] Quoc Tran
\item[] Aaron Wolen
\item[]
\item[] NIH/NIGMS
  \ei
\end{column}
\end{columns}

\end{frame}


\begin{frame}[c]{}

\Large

Slides: \href{https://bit.ly/rstudio2019}{\tt bit.ly/rstudio2019} \quad
\includegraphics[height=5mm]{Figs/cc-zero.png}

\vspace{7mm}

\href{https://kbroman.org}{\tt \lolit kbroman.org}

\vspace{7mm}

\href{https://kbroman.org/qtl2}{\tt kbroman.org/qtl2}

\vspace{7mm}

\href{https://github.com/kbroman}{\tt \lolit github.com/kbroman}

\vspace{7mm}

\href{https://twitter.com/kwbroman}{\tt \lolit @kwbroman}


\end{frame}




\end{document}
